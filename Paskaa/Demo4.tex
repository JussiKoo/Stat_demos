% Options for packages loaded elsewhere
\PassOptionsToPackage{unicode}{hyperref}
\PassOptionsToPackage{hyphens}{url}
%
\documentclass[
]{article}
\usepackage{amsmath,amssymb}
\usepackage{iftex}
\ifPDFTeX
  \usepackage[T1]{fontenc}
  \usepackage[utf8]{inputenc}
  \usepackage{textcomp} % provide euro and other symbols
\else % if luatex or xetex
  \usepackage{unicode-math} % this also loads fontspec
  \defaultfontfeatures{Scale=MatchLowercase}
  \defaultfontfeatures[\rmfamily]{Ligatures=TeX,Scale=1}
\fi
\usepackage{lmodern}
\ifPDFTeX\else
  % xetex/luatex font selection
\fi
% Use upquote if available, for straight quotes in verbatim environments
\IfFileExists{upquote.sty}{\usepackage{upquote}}{}
\IfFileExists{microtype.sty}{% use microtype if available
  \usepackage[]{microtype}
  \UseMicrotypeSet[protrusion]{basicmath} % disable protrusion for tt fonts
}{}
\makeatletter
\@ifundefined{KOMAClassName}{% if non-KOMA class
  \IfFileExists{parskip.sty}{%
    \usepackage{parskip}
  }{% else
    \setlength{\parindent}{0pt}
    \setlength{\parskip}{6pt plus 2pt minus 1pt}}
}{% if KOMA class
  \KOMAoptions{parskip=half}}
\makeatother
\usepackage{xcolor}
\usepackage[margin=1in]{geometry}
\usepackage{color}
\usepackage{fancyvrb}
\newcommand{\VerbBar}{|}
\newcommand{\VERB}{\Verb[commandchars=\\\{\}]}
\DefineVerbatimEnvironment{Highlighting}{Verbatim}{commandchars=\\\{\}}
% Add ',fontsize=\small' for more characters per line
\usepackage{framed}
\definecolor{shadecolor}{RGB}{248,248,248}
\newenvironment{Shaded}{\begin{snugshade}}{\end{snugshade}}
\newcommand{\AlertTok}[1]{\textcolor[rgb]{0.94,0.16,0.16}{#1}}
\newcommand{\AnnotationTok}[1]{\textcolor[rgb]{0.56,0.35,0.01}{\textbf{\textit{#1}}}}
\newcommand{\AttributeTok}[1]{\textcolor[rgb]{0.13,0.29,0.53}{#1}}
\newcommand{\BaseNTok}[1]{\textcolor[rgb]{0.00,0.00,0.81}{#1}}
\newcommand{\BuiltInTok}[1]{#1}
\newcommand{\CharTok}[1]{\textcolor[rgb]{0.31,0.60,0.02}{#1}}
\newcommand{\CommentTok}[1]{\textcolor[rgb]{0.56,0.35,0.01}{\textit{#1}}}
\newcommand{\CommentVarTok}[1]{\textcolor[rgb]{0.56,0.35,0.01}{\textbf{\textit{#1}}}}
\newcommand{\ConstantTok}[1]{\textcolor[rgb]{0.56,0.35,0.01}{#1}}
\newcommand{\ControlFlowTok}[1]{\textcolor[rgb]{0.13,0.29,0.53}{\textbf{#1}}}
\newcommand{\DataTypeTok}[1]{\textcolor[rgb]{0.13,0.29,0.53}{#1}}
\newcommand{\DecValTok}[1]{\textcolor[rgb]{0.00,0.00,0.81}{#1}}
\newcommand{\DocumentationTok}[1]{\textcolor[rgb]{0.56,0.35,0.01}{\textbf{\textit{#1}}}}
\newcommand{\ErrorTok}[1]{\textcolor[rgb]{0.64,0.00,0.00}{\textbf{#1}}}
\newcommand{\ExtensionTok}[1]{#1}
\newcommand{\FloatTok}[1]{\textcolor[rgb]{0.00,0.00,0.81}{#1}}
\newcommand{\FunctionTok}[1]{\textcolor[rgb]{0.13,0.29,0.53}{\textbf{#1}}}
\newcommand{\ImportTok}[1]{#1}
\newcommand{\InformationTok}[1]{\textcolor[rgb]{0.56,0.35,0.01}{\textbf{\textit{#1}}}}
\newcommand{\KeywordTok}[1]{\textcolor[rgb]{0.13,0.29,0.53}{\textbf{#1}}}
\newcommand{\NormalTok}[1]{#1}
\newcommand{\OperatorTok}[1]{\textcolor[rgb]{0.81,0.36,0.00}{\textbf{#1}}}
\newcommand{\OtherTok}[1]{\textcolor[rgb]{0.56,0.35,0.01}{#1}}
\newcommand{\PreprocessorTok}[1]{\textcolor[rgb]{0.56,0.35,0.01}{\textit{#1}}}
\newcommand{\RegionMarkerTok}[1]{#1}
\newcommand{\SpecialCharTok}[1]{\textcolor[rgb]{0.81,0.36,0.00}{\textbf{#1}}}
\newcommand{\SpecialStringTok}[1]{\textcolor[rgb]{0.31,0.60,0.02}{#1}}
\newcommand{\StringTok}[1]{\textcolor[rgb]{0.31,0.60,0.02}{#1}}
\newcommand{\VariableTok}[1]{\textcolor[rgb]{0.00,0.00,0.00}{#1}}
\newcommand{\VerbatimStringTok}[1]{\textcolor[rgb]{0.31,0.60,0.02}{#1}}
\newcommand{\WarningTok}[1]{\textcolor[rgb]{0.56,0.35,0.01}{\textbf{\textit{#1}}}}
\usepackage{graphicx}
\makeatletter
\def\maxwidth{\ifdim\Gin@nat@width>\linewidth\linewidth\else\Gin@nat@width\fi}
\def\maxheight{\ifdim\Gin@nat@height>\textheight\textheight\else\Gin@nat@height\fi}
\makeatother
% Scale images if necessary, so that they will not overflow the page
% margins by default, and it is still possible to overwrite the defaults
% using explicit options in \includegraphics[width, height, ...]{}
\setkeys{Gin}{width=\maxwidth,height=\maxheight,keepaspectratio}
% Set default figure placement to htbp
\makeatletter
\def\fps@figure{htbp}
\makeatother
\setlength{\emergencystretch}{3em} % prevent overfull lines
\providecommand{\tightlist}{%
  \setlength{\itemsep}{0pt}\setlength{\parskip}{0pt}}
\setcounter{secnumdepth}{-\maxdimen} % remove section numbering
\ifLuaTeX
  \usepackage{selnolig}  % disable illegal ligatures
\fi
\IfFileExists{bookmark.sty}{\usepackage{bookmark}}{\usepackage{hyperref}}
\IfFileExists{xurl.sty}{\usepackage{xurl}}{} % add URL line breaks if available
\urlstyle{same}
\hypersetup{
  pdftitle={Demo 4},
  pdfauthor={Jussi Kauppinen},
  hidelinks,
  pdfcreator={LaTeX via pandoc}}

\title{Demo 4}
\author{Jussi Kauppinen}
\date{2024-04-20}

\begin{document}
\maketitle

\hypertarget{task-4.1}{%
\subsection{Task 4.1}\label{task-4.1}}

\begin{enumerate}
\def\labelenumi{\alph{enumi})}
\tightlist
\item
\end{enumerate}

\begin{equation}
\begin{split}
L(\mu, \sigma^2) &= \left(\frac{1}{\sqrt{2\pi\sigma^2}}\right)^n exp(-\frac{1}{2\sigma^2} \sum_{i=1}^n (y_i - \mu)^2)\\
l(\mu, \sigma^2) &= n \log \left( \frac{1}{\sqrt{2\pi\sigma^2}} \right) - \frac{1}{2\sigma^2} \sum_{i=1}^n (y_i - \mu)^2\\
S(\mu) &= - \frac{1}{2\sigma^2} \sum_{i=1}^n \frac{\partial}{\partial \mu} (y_i - \mu)^2\\
&= - \frac{1}{2\sigma^2} \sum_{i=1}^n -2(y_i - \mu)\\
&= \frac{1}{\sigma^2} \sum_{i=1}^n (y_i - \mu)\\
\end{split}
\end{equation}

\begin{enumerate}
\def\labelenumi{\alph{enumi})}
\setcounter{enumi}{1}
\tightlist
\item
\end{enumerate}

\begin{equation}
\begin{split}
L(\lambda) &= e^{-\lambda n} \frac{\lambda^{\sum_{i=1}^n y_i}}{\prod_{i=1}^n y_i !}\\
l(\lambda) &= -\lambda n + \log \lambda \sum_{i=1}^n y_i - \sum_{i=1}^n \log y_i !\\
S(\lambda) &= \frac{1}{\lambda} \sum_{i=1}^n y_i - n\\
\end{split}
\end{equation}

\hypertarget{task-4.2}{%
\subsection{Task 4.2}\label{task-4.2}}

\begin{Shaded}
\begin{Highlighting}[]
\CommentTok{\#a}

\CommentTok{\# Score function for the normal distribution}
\CommentTok{\# where the root should be at the origin}
\NormalTok{ScN }\OtherTok{\textless{}{-}} \ControlFlowTok{function}\NormalTok{(y, mu, n) \{ n }\SpecialCharTok{*}\NormalTok{ (}\FunctionTok{mean}\NormalTok{(y) }\SpecialCharTok{{-}}\NormalTok{ mu) \}}


\NormalTok{plotScoresN }\OtherTok{\textless{}{-}} \ControlFlowTok{function}\NormalTok{(n)\{}
\NormalTok{  y }\OtherTok{\textless{}{-}} \FunctionTok{rnorm}\NormalTok{(n, }\DecValTok{0}\NormalTok{, }\DecValTok{1}\NormalTok{)}
\NormalTok{  mu }\OtherTok{\textless{}{-}} \FunctionTok{seq}\NormalTok{(}\SpecialCharTok{{-}}\DecValTok{2}\NormalTok{, }\DecValTok{2}\NormalTok{, }\FloatTok{0.1}\NormalTok{)}
  \FunctionTok{plot}\NormalTok{(mu, }\FunctionTok{ScN}\NormalTok{(y, mu, n), }\AttributeTok{type =} \StringTok{"l"}\NormalTok{, }\AttributeTok{xlab =} \StringTok{"mu"}\NormalTok{, }\AttributeTok{ylab =} \StringTok{"S(mu)"}\NormalTok{)}
  \FunctionTok{abline}\NormalTok{(}\DecValTok{0}\NormalTok{,}\DecValTok{0}\NormalTok{)}
  \ControlFlowTok{for}\NormalTok{ (i }\ControlFlowTok{in} \DecValTok{1}\SpecialCharTok{:}\DecValTok{19}\NormalTok{) \{}
\NormalTok{  y }\OtherTok{\textless{}{-}} \FunctionTok{rnorm}\NormalTok{(n, }\DecValTok{0}\NormalTok{, }\DecValTok{1}\NormalTok{)}
  \FunctionTok{lines}\NormalTok{(mu, }\FunctionTok{ScN}\NormalTok{(y, mu, n))}
\NormalTok{  \}}
\NormalTok{\}}

\FunctionTok{par}\NormalTok{(}\AttributeTok{mfrow =} \FunctionTok{c}\NormalTok{(}\DecValTok{2}\NormalTok{, }\DecValTok{2}\NormalTok{))}
\FunctionTok{plotScoresN}\NormalTok{(}\DecValTok{10}\NormalTok{)}
\FunctionTok{plotScoresN}\NormalTok{(}\DecValTok{100}\NormalTok{)}
\FunctionTok{plotScoresN}\NormalTok{(}\DecValTok{1000}\NormalTok{)}
\FunctionTok{plotScoresN}\NormalTok{(}\DecValTok{10000}\NormalTok{)}
\end{Highlighting}
\end{Shaded}

\includegraphics{Demo4_files/figure-latex/unnamed-chunk-1-1.pdf}

\begin{Shaded}
\begin{Highlighting}[]
\CommentTok{\#b}

\NormalTok{ScP }\OtherTok{\textless{}{-}} \ControlFlowTok{function}\NormalTok{(y, lambda, n) \{ n }\SpecialCharTok{*}\NormalTok{ (}\FunctionTok{mean}\NormalTok{(y)}\SpecialCharTok{/}\NormalTok{lambda }\SpecialCharTok{{-}} \DecValTok{1}\NormalTok{) \}}

\NormalTok{plotScoresP }\OtherTok{\textless{}{-}} \ControlFlowTok{function}\NormalTok{(n)\{}
\NormalTok{  y }\OtherTok{\textless{}{-}} \FunctionTok{rpois}\NormalTok{(n, }\DecValTok{4}\NormalTok{)}
\NormalTok{  lambda }\OtherTok{\textless{}{-}} \FunctionTok{seq}\NormalTok{(}\DecValTok{2}\NormalTok{, }\DecValTok{6}\NormalTok{, }\FloatTok{0.1}\NormalTok{)}
  \FunctionTok{plot}\NormalTok{(lambda, }\FunctionTok{ScP}\NormalTok{(y, lambda, n), }\AttributeTok{type =} \StringTok{"l"}\NormalTok{, }\AttributeTok{xlab =} \StringTok{"lambda"}\NormalTok{, }\AttributeTok{ylab =} \StringTok{"S(lambda)"}\NormalTok{)}
  \FunctionTok{abline}\NormalTok{(}\DecValTok{0}\NormalTok{,}\DecValTok{0}\NormalTok{)}
  \ControlFlowTok{for}\NormalTok{ (i }\ControlFlowTok{in} \DecValTok{1}\SpecialCharTok{:}\DecValTok{19}\NormalTok{) \{}
\NormalTok{  y }\OtherTok{\textless{}{-}} \FunctionTok{rpois}\NormalTok{(n, }\DecValTok{4}\NormalTok{)}
  \FunctionTok{lines}\NormalTok{(lambda, }\FunctionTok{ScP}\NormalTok{(y, lambda, n))}
\NormalTok{  \}}
\NormalTok{\}}

\FunctionTok{par}\NormalTok{(}\AttributeTok{mfrow =} \FunctionTok{c}\NormalTok{(}\DecValTok{2}\NormalTok{, }\DecValTok{2}\NormalTok{))}
\FunctionTok{plotScoresP}\NormalTok{(}\DecValTok{10}\NormalTok{)}
\FunctionTok{plotScoresP}\NormalTok{(}\DecValTok{100}\NormalTok{)}
\FunctionTok{plotScoresP}\NormalTok{(}\DecValTok{1000}\NormalTok{)}
\FunctionTok{plotScoresP}\NormalTok{(}\DecValTok{10000}\NormalTok{)}
\end{Highlighting}
\end{Shaded}

\includegraphics{Demo4_files/figure-latex/unnamed-chunk-1-2.pdf} As the
sample size gets larger the score functions values vary less. This can
be explained by central limit theorem since both score functions here
include the mean of the samples.

\hypertarget{task-4.3}{%
\subsection{Task 4.3}\label{task-4.3}}

\begin{Shaded}
\begin{Highlighting}[]
\NormalTok{densP }\OtherTok{\textless{}{-}} \ControlFlowTok{function}\NormalTok{(n)\{}
\NormalTok{  scores }\OtherTok{\textless{}{-}} \FunctionTok{numeric}\NormalTok{(}\DecValTok{100}\NormalTok{)}
  \ControlFlowTok{for}\NormalTok{ (i }\ControlFlowTok{in} \DecValTok{1}\SpecialCharTok{:}\DecValTok{100}\NormalTok{)\{}
\NormalTok{    y }\OtherTok{\textless{}{-}} \FunctionTok{rpois}\NormalTok{(n, }\DecValTok{4}\NormalTok{)}
\NormalTok{    scores[i] }\OtherTok{\textless{}{-}} \FunctionTok{ScP}\NormalTok{(y, }\DecValTok{4}\NormalTok{, n)}
\NormalTok{  \}}
  
  \FunctionTok{plot}\NormalTok{(}\FunctionTok{density}\NormalTok{(scores), }\AttributeTok{main=}\FunctionTok{paste}\NormalTok{(}\StringTok{"Sample size: "}\NormalTok{, n))}
\NormalTok{\}}

\FunctionTok{par}\NormalTok{(}\AttributeTok{mfrow =} \FunctionTok{c}\NormalTok{(}\DecValTok{2}\NormalTok{, }\DecValTok{2}\NormalTok{))}
\FunctionTok{densP}\NormalTok{(}\DecValTok{10}\NormalTok{)}
\FunctionTok{densP}\NormalTok{(}\DecValTok{100}\NormalTok{)}
\FunctionTok{densP}\NormalTok{(}\DecValTok{1000}\NormalTok{)}
\FunctionTok{densP}\NormalTok{(}\DecValTok{10000}\NormalTok{)}
\end{Highlighting}
\end{Shaded}

\includegraphics{Demo4_files/figure-latex/unnamed-chunk-2-1.pdf}

We notice from the scale of the y-axis that the density plot gets
flatter as sample size gets larger. One could also say that score
functions values distribute more evenly.

\hypertarget{task-4.4}{%
\subsection{Task 4.4}\label{task-4.4}}

\begin{enumerate}
\def\labelenumi{\alph{enumi})}
\tightlist
\item
\end{enumerate}

Score test statistic is \(\frac{S}{\sqrt{I}}\).

\begin{equation}
\begin{split}
I &= -E\left[ \frac{\partial}{\partial \mu} S(\mu)\right]\\
&= -E\left[ \frac{\partial}{\partial \mu} \frac{1}{\sigma^2} \sum_{i=1}^n (y_i - \mu)\right]\\
&= -E\left[ -\frac{n}{\sigma^2}\right]\\
&= \frac{n}{\sigma^2}
\end{split}
\end{equation}

So the test statistic is

\begin{equation}
\begin{split}
\frac{S}{\sqrt{I}} = \frac{\frac{1}{\sigma^2} \sum_{i=1}^n (y_i - \mu)}{\sqrt{\frac{n}{\sigma^2}}} = \frac{1}{\sqrt{\sigma^2 n}} \sum_{i=1}^n (y_i - \mu)\\
\end{split}
\end{equation}

\begin{enumerate}
\def\labelenumi{\alph{enumi})}
\setcounter{enumi}{1}
\tightlist
\item
\end{enumerate}

\begin{equation}
\begin{split}
I &= -E\left[ \frac{\partial}{\partial \lambda} S(\lambda)\right]\\
&= -E\left[ \frac{\partial}{\partial \lambda} \frac{1}{\lambda} \sum_{i=1}^n y_i - n\right]\\
&= -E\left[-\frac{1}{\lambda^2} \sum_{i=1}^n y_i\right]\\
&= \frac{1}{\lambda^2} \sum_{i=1}^n E[y_i] = \frac{1}{\lambda^2} \cdot n\lambda = \frac{n}{\lambda}\\
\end{split}
\end{equation}

So the test statistic is

\begin{equation}
\begin{split}
\frac{S}{\sqrt{I}} = \frac{\frac{1}{\lambda} \sum_{i=1}^n y_i - n}{\sqrt{\frac{n}{\lambda}}} = \frac{1}{\sqrt{\lambda n}} \sum_{i=1}^n y_i - \sqrt{\lambda n}\\
\end{split}
\end{equation}

\hypertarget{task-4.5}{%
\subsection{Task 4.5}\label{task-4.5}}

\begin{equation}
\begin{split}
D &= 2\log \Lambda = 2l(\hat\theta_{max}) - 2l(\hat\theta)\\
&= 2\sum_{i=1}^n \frac{y_i \hat\theta_{i,max} - b(\hat\theta_{i,max})}{a_i(\phi)} + 2\sum_{i=1}^n c(y_i, \phi) - 2\sum_{i=1}^n \frac{y_i \hat\theta_{i} - b(\hat\theta_{i})}{a_i(\phi)} - 2\sum_{i=1}^n c(y_i, \phi)\\
&= 2\left( \sum_{i=1}^n \frac{y_i (\hat\theta_{i,max} - \hat\theta_{i}) - b(\hat\theta_{i,max}) + b(\hat\theta_{i})}{a_i(\phi)}  \right)\\
\end{split}
\end{equation}

\hypertarget{task-4.6}{%
\subsection{Task 4.6}\label{task-4.6}}

\begin{Shaded}
\begin{Highlighting}[]
\NormalTok{ei }\OtherTok{\textless{}{-}} \FunctionTok{read.table}\NormalTok{(}\StringTok{"http://users.jyu.fi/\textasciitilde{}knordhau/GLM2/korvatul.txt"}\NormalTok{, }\AttributeTok{header=}\ConstantTok{TRUE}\NormalTok{)}
\end{Highlighting}
\end{Shaded}

\begin{enumerate}
\def\labelenumi{\alph{enumi})}
\tightlist
\item
\end{enumerate}

\end{document}
